\section{Architektur}
Die Architektur beschreibt den allgemeinen Aufbau des Betriebssystems. Die genauen Beschreibungen zu den einzelnen Teilen sind in weiteren Kapiteln in diesem Dokument zu finden.

\subsection{Art des Kernel}
Das Betriebssystem ist ein Monolithischer Kernel. 

\subsection{Vorgegebene Anforderungen an das Betriebssystem}
Im folgenden befindet sich eine Auflistung mit vorgegebenen Anforderungen an das Betriebssystem.

\begin{table}[H]
\begin{tabular}{ p{5cm}| p{9cm} }
  \textbf{Anforderung} & \textbf{Erklärung} \\ 
  \hline
  Single-User & Das Betriebssystem muss zu jedem Zeitpunkt nur einen Benutzer bzw. eine Benutzerin händeln. \\
  Lauffähige Anwendung & Auf dem Betriebssystem muss sich eine lauffähige Anwendung befinden. \\
  Präemptives Multitasking & Das Betriebssystem kann gleichzeitig mehrere Prozesse ausführen.\\
  Konsole & Es muss ein Konsole zur Kommunikation mit dem Betriebssystem geben. \\
  Interprozess-Kommunikation & Das Betriebssystem muss Möglichkeit zu Kommunikation zwischen Prozessen zur Verfügung stellen. \\
  Sicherheit & Es muss eine strikte Trennung zwischen User- und Systemmode vorhanden sein. \\
  Robustheit & Das Betriebssystem darf von Programmabstürzen nicht beeinflusst werden. \\
  Virtueller Speicher & Memory Management muss für größere Anwendungen vorhanden sein. \\
  SD-Karte & Externe Anwendungen sollten von der SD-Karte geladen werden können. \\
  Dateisystem & Das Betriebssystem muss ein Dateisystem (FAT oer FAT32) besitzen. \\
  Portierbarkeit & Für eine Portierbarkeit des Systems muss ein HAL umgesetzt werden. \\
  Integration von Geräten & Das Betriebssystem muss eine einfache Integration von verschiedenen Geräten gewährleisten. \\
  Performanztests & Es müssen Performanztests zur Leistungsfeststellung des Systems durchgeführt werden. \\  
 \end{tabular}
 \caption{Vorgegebene Anforderungen}
 \label{table:Prescribed-Requirements}
\end{table}

\subsection{Eigene Anforderungen an das Betriebssystem}

\pagebreak 