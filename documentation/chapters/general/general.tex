\section{Allgemein}
Allgemeine Aspekte zum Betriebssystem werden in diesem Kapitel erläutert. Dazu zählen vorgegebene Anforderungen, eigene Anforderungen und das Ergebnis in Zusammenhang mit den zuvor definierten Anforderungen.

\subsection{Vorgegebene Anforderungen an das Betriebssystem}
Eine Auflistung aller vorgegebenen Anforderungen an das Betriebssystem sind in Tabelle \ref{table:Prescribed-Requirements} angegeben.

\begin{table}[H]
\begin{tabular}{ p{5cm}| p{9cm} }
  \textbf{Anforderung} & \textbf{Erklärung} \\ 
  \hline
  Single-User & Das Betriebssystem muss zu jedem Zeitpunkt nur einen Benutzer bzw. eine Benutzerin händeln. \\
  Lauffähige Anwendung & Auf dem Betriebssystem muss sich eine lauffähige Anwendung befinden. \\
  Präemptives Multitasking & Das Betriebssystem kann gleichzeitig mehrere Prozesse ausführen.\\
  Konsole & Es muss ein Konsole zur Kommunikation mit dem Betriebssystem geben. \\
  Interprozess-Kommunikation & Das Betriebssystem muss Möglichkeit zu Kommunikation zwischen Prozessen zur Verfügung stellen. \\
  Sicherheit & Es muss eine strikte Trennung zwischen User- und Systemmode vorhanden sein. \\
  Robustheit & Das Betriebssystem darf von Programmabstürzen nicht beeinflusst werden. \\
  Virtueller Speicher & Memory Management muss für größere Anwendungen vorhanden sein. \\
  SD-Karte & Externe Anwendungen sollten von der SD-Karte geladen werden können. \\
  Dateisystem & Das Betriebssystem muss ein Dateisystem (FAT oer FAT32) besitzen. \\
  Portierbarkeit & Für eine Portierbarkeit des Systems muss ein HAL umgesetzt werden. \\
  Integration von Geräten & Das Betriebssystem muss eine einfache Integration von verschiedenen Geräten gewährleisten. \\
  Performanztests & Es müssen Performanztests zur Leistungsfeststellung des Systems durchgeführt werden. \\  
 \end{tabular}
 \caption{Vorgegebene Anforderungen}
 \label{table:Prescribed-Requirements}
\end{table}

\subsection{Eigene Anforderungen an das Betriebssystem}
Eine Auflistung aller eigenen Anforderungen an das Betriebssystem sind in Tabelle \ref{table:Own-Requirements} angegeben.

\begin{table}[H]
\begin{tabular}{ p{5cm}| p{9cm} }
  \textbf{Anforderung} & \textbf{Erklärung} \\ 
  \hline
  Hoher Abstraktionsgrad & Alle Komponenten sollten voneinander abstrahiert sein. \\
  Intuitiver Aufbau & Die Komponenten des Betriebssystem sollten möglichst selbsterklärend sein. \\
  Leichte Erweiterbarkeit & Laufende Erweiterungen sollten ohne große Veränderungen umgesetzt werden können. \\
  Einfache Wartung & Das Betriebssystem sollte so Aufgebaut sein, dass eventuelle Ausbesserungen am Betriebssystem nur an einer dafür zuständigen Komponente vorgenommen werden müssen. \\
 \end{tabular}
 \caption{Vorgegebene Anforderungen}
 \label{table:Own-Requirements}
\end{table}

\subsection{Resultat des Betriebssystems}
Im Allgemeinen wurden alle zuvor erwähnten Anforderungen an das Betriebssystem erfüllt. Einzelne Verbesserungs- bzw. Erweiterungsvorschläge können aus Kapitel XXX - Ausblick entnommen werden. \\
Stabilitätstests, welche während dem Projekt durchgeführt wurden, haben aufgezeigt, dass das System über zwölf Stunden fehlerlos durchläuft.

\pagebreak 