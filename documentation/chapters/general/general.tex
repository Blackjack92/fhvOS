\section{Allgemein}
In diesem Kapitel werden allgemeine Aspekte zum Betriebsystem erläutert. Dazu zählen insbesondere die durch das Studienprojekt definierten Anforderungen, zusätzlich durch die Studierenden gesetzte Anforderungen und die erreichten Ergebnisse hinsichtlich dieser Anforderungen.

\subsection{Vorgegebene Anforderungen an das Betriebssystem}
Eine Auflistung aller vorgegebenen Anforderungen, insbesondere funktionale Anforderungen, an das Betriebssystem sind in Tabelle \ref{table:Prescribed-Requirements} angegeben.

\begin{table}[H]
\begin{tabular}{ p{5cm}| p{9cm} }
  \textbf{Anforderung} & \textbf{Erklärung} \\ 
  \hline
  Single-User & Das Betriebssystem muss zu jedem Zeitpunkt nur eine Benutzerin bzw. einen Benutzer verwalten. \\
  Lauffähige Anwendung & Auf dem Betriebssystem muss zumindest eine lauffähige Anwendung ausführbar sein. \\
  Präemptives Multitasking & Das Betriebssystem muss gleichzeitig mehrere Prozesse ausführen können, wobei für jeden Prozess eine bestimmte Zeitscheibe vorgesehen ist.\\
  Konsole & Es muss ein Konsole zum Absetzen von Befehlen vorhanden sein. \\
  Interprozess-Kommunikation & Das Betriebssystem muss eine Möglichkeit zur Kommunikation zwischen Prozessen zur Verfügung stellen. \\
  Sicherheit & Es muss eine strikte Trennung zwischen User- und Systemmodus vorhanden sein. \\
  Robustheit & Das Betriebssystem, respektive dessen Stabilität, darf von Programmabstürzen nicht beeinflusst werden. \\
  Virtueller Speicher & \textit{Memory Management} muss für größere Anwendungen vorhanden sein. \\
  SD-Karte & Externe Anwendungen sollen von der SD-Karte nachgeladen werden können. \\
  Dateisystem & Das Betriebssystem muss ein beliebiges Dateisystem verwalten können.\\
  Portierbarkeit & Für eine Portierbarkeit des Systems muss ein \ac{HAL} umgesetzt werden. \\
  Integration von Geräten & Das Betriebssystem muss eine einfache Integration von verschiedenen Geräten gewährleisten. \\
  Performanztests & Es müssen Performanztests zur Leistungsfeststellung des Systems durchgeführt werden. \\  
 \end{tabular}
 \caption{Vorgegebene Anforderungen}
 \label{table:Prescribed-Requirements}
\end{table}

\subsection{Eigene Anforderungen an das Betriebssystem}
Zusätzlich zu den oben angeführten Anforderungen wurden weitere, nicht-funktionale Anforderungen an das Betriebssystem, durch die an der Entwicklung beteiligten Studierenden definiert. Eine Auflistung aller eigenen Anforderungen an das Betriebssystem sind in Tabelle \ref{table:Own-Requirements} angegeben.

\begin{table}[H]
\begin{tabular}{ p{5cm}| p{9cm} }
  \textbf{Anforderung} & \textbf{Erklärung} \\ 
  \hline
  Hoher Abstraktionsgrad & Alle Komponenten des Betriebssystems sollen einen hohen Abstraktionsgrad aufweisen. \\
  Intuitiver Aufbau & Die Komponenten des Betriebssystem sollen eine intuitive Programmierschnittstelle aufweisen. \\
  Leichte Erweiterbarkeit & Mögliche Erweiterungen sollen ohne große Veränderungen an der Architektur umgesetzt werden können. \\
  Einfache Wartung & Das Betriebssystem soll eine einfache Wartbarkeit hinsichtlich Fehlern aufweisen. \\
  \textit{Moving Head} mit \ac{DMX} & Auf dem Betriebssystem soll eine Anwendung zur Ansteuerung eines oder mehrerer \textit{Moving Heads} mittels \ac{DMX}-Protokoll ausführbar sein. \\
 \end{tabular}
 \caption{Vorgegebene Anforderungen}
 \label{table:Own-Requirements}
\end{table}

\subsection{Erfüllung der Anforderungen}
Im Allgemeinen wurden alle zuvor erwähnten funktionalen Anforderungen an das Betriebssystem erfüllt. Einzelne Verbesserungs- bzw. Erweiterungsmöglichkeiten können aus Kapitel \ref{summary} werden. \\
Die Performanz des Betriebssystems wird in Kapitel \ref{Performanz} dokumentiert.
Hinsichtlich der Stabilität wurden keine konkreten Experimente durchgeführt, allerdings haben verschiedene Benutzungstests gezeigt, dass das Betriebssystem über mehrere Stunden ohne Abstürze lauffähig ist.

\pagebreak 