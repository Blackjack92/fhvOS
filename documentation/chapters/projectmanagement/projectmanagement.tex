\section{Projektmanagement}
Im folgenden Abschnitt werden einige Punkte zum Projektmanagement erklärt. 

\subsection{Prozessmodell}
Als Prozessmodell wurde SCRUM mit einigen Abänderungen umgesetzt. Wobei das zentrale Vorgehen in Bezug auf Agilität bestmöglich übernommen wurde. Gründe für die Verwendung von SCRUM:

\begin{itemize}
	\item Agilität
	\item Nach jedem Sprint ein lauffähiges System
	\item Klares Ziel für jeden Sprint
	\item Einfaches Hinzufügen fehlender Aufgaben (neue Stories)
	\item Einfaches neu priorisieren von Aufgaben
	\item Ereignisse die den Entwicklungszyklus beeinflussen (z.B. Klausuren) beeinträchtigen das weitere Vorgehen nicht
	\item Klare Übersicht der fehlenden Stories
\end{itemize}

Alle Aufgaben wurden im Repository als Issues aufgenommen und können unter folgendem Link eingesehen werden:

\url{https://github.com/Blackjack92/fhvOS/issues}
\linebreak

Insgesamt wurden mehr als einhundert Issues aufgenommen und davon über neunzig Prozent gelöst. Alle offenen Punkte sind mit einer niederen Priorität eingestuft und beeinflussen die korrekte Funktionsfähigkeit des Betriebssystem nicht.

\subsection{Versionsverwaltung}
Als Versionsverwaltung wurde Git verwendet. Zwei Gründe für die Verwendung von Git sind leichte Einbindung im Zusammenhang mit dem Repository (siehe \ref{Repository}) und die Möglichkeit der Nicht-linearen Entwicklung.

\subsection{Repository}
Das Repository wurde auf Github angelegt. Das Repository wurde veröffentlicht, da dies eine Voraussetzung für die kostenlose Nutzung von Github ist. Unter dem folgenden Link kann das Projekt eingesehen werden:

\url{https://github.com/Blackjack92/fhvOS}
\linebreak

\subsection{Zeitplan}
Der Zeitplan des Projekts wurde mittels Microsoft Project erstellt und wurde während des gesamten Projekts, bis auf wenige Ausnahmen, eingehalten. In den beiliegenden Unterlagen ist der Zeitplan unter der Bezeichnung XXX zu finden.

\pagebreak 