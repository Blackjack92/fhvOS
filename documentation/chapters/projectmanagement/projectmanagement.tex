\section{Projektmanagement}
Im folgenden Abschnitt werden einige Punkte zum Projektmanagement erklärt. 

\subsection{Prozessmodell}
Als Prozessmodell wurde SCRUM mit einigen Abänderungen umgesetzt. Wobei das zentrale Vorgehen in Bezug auf Agilität bestmöglich übernommen wurde. Gründe für die Verwendung von SCRUM:

\begin{itemize}
	\item Agilität
	\item Nach jedem Sprint ein lauffähiges System
	\item Klares Ziel für jeden Sprint
	\item Einfaches Hinzufügen fehlender Aufgaben (neue Stories)
	\item Einfaches neu priorisieren von Aufgaben
	\item Ereignisse die den Entwicklungszyklus beeinflussen (z.B. Klausuren) beeinträchtigen das weitere Vorgehen nicht
	\item Klare Übersicht der fehlenden Stories
\end{itemize}

\subsection{Versionsverwaltung}
Als Versionsverwaltung wurde Git verwendet. Zwei Gründe für die Verwendung von Git sind leichte Einbindung im Zusammenhang mit dem Repository (siehe \ref{Repository}) und die Möglichkeit der Nicht-linearen Entwicklung.

\subsection{Repository}
Github.

\pagebreak 