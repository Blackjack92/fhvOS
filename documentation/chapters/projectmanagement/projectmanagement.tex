\section{Projektmanagement}
Im folgenden Abschnitt genauer auf das Projektmanagement eingegangen. Die zentrale Punkte in diesem Kapitel sind: das Prozessmodell, die Versionsverwaltung, das Repository sowie der Zeitplan.

\subsection{Prozessmodell}
Als Prozessmodell wurde SCRUM mit einigen Abänderungen umgesetzt. Wobei das zentrale Vorgehen in Bezug auf Agilität bestmöglich übernommen wurde. Gründe für die Verwendung von SCRUM sind: Agilität, nach jedem Sprint ein lauffähiges System, klares Ziel für jeden Sprint, einfaches Hinzufügen fehlender Aufgaben (neue Stories), einfaches neu priorisieren von Aufgaben, Ereignisse die den Entwicklungszyklus beeinflussen (z.B. Klausuren) beeinträchtigen das weitere Vorgehen nicht, klare Übersicht der fehlenden Stories.

Alle Aufgaben wurden im Repository als Issues aufgenommen und können unter folgendem Link eingesehen werden: \\

\url{https://github.com/Blackjack92/fhvOS/issues} \\

Insgesamt wurden mehr als einhundert Issues aufgenommen und davon über neunzig Prozent gelöst. Alle offenen Punkte sind mit einer niederen Priorität eingestuft und beeinflussen die korrekte Funktionsfähigkeit des Betriebssystem nicht.

\subsection{Versionsverwaltung}
Als Versionsverwaltung wurde Git verwendet. Zwei Gründe für die Verwendung von Git sind leichte Einbindung im Zusammenhang mit dem Repository (siehe \ref{Repository}) und die Möglichkeit der Nicht-linearen Entwicklung.

\subsection{Repository}
Das Repository wurde auf Github angelegt und veröffentlicht. Das Veröffentlichen war nötig, da dies eine Voraussetzung für die kostenlose Nutzung von Github ist. Unter dem folgenden Link kann das Projekt eingesehen werden: \\

\url{https://github.com/Blackjack92/fhvOS} \\

\subsection{Zeitplan}
Der Zeitplan des Projekts wurde mittels Microsoft Project erstellt und wurde während des gesamten Projekts, bis auf wenige Ausnahmen, eingehalten. In den beiliegenden Unterlagen ist der Zeitplan unter der Bezeichnung XXX zu finden.

\pagebreak 