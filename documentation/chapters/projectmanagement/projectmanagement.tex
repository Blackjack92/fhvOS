\section{Projektmanagement}
Im folgenden Abschnitt wird das Projektmanagment und das verwendete Prozessmodell beschrieben. Weiters sind hier die Zugänge zum \textit{Repository} und dem Ticketsystem dokumentiert.

\subsection{Prozessmodell}
Als Prozessmodell wurde \textit{SCRUM}, allerdings mit einigen Adaptionen umgesetzt, wobei das zentrale Vorgehen in Bezug auf Agilität bestmöglich übernommen wurde. Gründe für die Verwendung von \textit{SCRUM} waren vor allem die Möglichkeit zur agilen Umsetzung der vorhandenen und neuer Anforderungen.\\
Es wurde auf das Konzept eines \textit{SCRUM}-Boards verzichtet, stattdessen wurden sämtliche \textit{Stories} als eigenes Ticket in einem passenden Ticketsystem angelegt. Es erfolgte eine Priorisierung der jeweiligen Tickets. Die Abarbeitungsreihenfolge dieser Tickets ergab sich schließlich nach der jeweiligen Priorität selbst.\\

Die angelegten Tickets finden sich unter folgendem Link:

\url{https://github.com/Blackjack92/fhvOS/issues} \\

Durch Einsicht der offenen und geschlossenen Tickets lässt sich sowohl der Entwicklungsfortschritt, als auch die jeweiligen Designentscheidungen sehr gut nachvollziehen.

\subsection{Versionsverwaltung}
Als Versionsverwaltung wurde \textit{Git} verwendet. Die Entscheidung für die Verwendung von \textit{Git} sind insbesondere die leichte Einbindung im Zusammenhang mit dem angelegten \textit{Repository} und die Möglichkeit der Nicht-linearen Entwicklung.

\subsection{Repository}
\label{Repository}
Das \textit{Repository} für den Source-Code und weitere relevante Dokumente für die Entwicklung des Betriebssystems, wurde auf \textit{Github} angelegt und veröffentlicht. Der Source-Code des Betriebssystems war während der gesamten Entwicklungszeit öffentlich zugänglich. Unter folgendem Link ist das \textit{Repository} einsehbar:\\

\url{https://github.com/Blackjack92/fhvOS} \\

\subsection{Zeitplan}
Der Zeitplan des Projekts wurde mittels \textit{Microsoft Project} erstellt und verwaltet. Der Zeitplan selbst wurde während des gesamten Projekt im Allgemeinen sehr gut eingehalten. Das angelegte \textit{Microsoft Project}-Projekt ist ebenfalls im oben beschriebenen \textit{Repository} im Ordner \textit{projectmanagement} einsehbar.

\pagebreak 