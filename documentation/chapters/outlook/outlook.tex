\section{Ausblick}
Diese Kapitel enthält befasst sich mit Inhalten, welche in diesem Betriebssystem nicht zur Gänze implementiert wurden bzw. noch Verbesserungspotential aufweisen.

\subsection{Offene Punkte aus den Anforderungen}
Das Betriebssystem schneidet alle Punkte, welche in den Anforderungen erwähnt wurden, zumindest teilweise an.

\subsection{Punkte mit Verbesserungspotential}
Einige Punkte des Betriebssystem konnten nicht vollständig abgedeckt werden. Diese Punkte mit Verbesserungspotential werden in Tabelle \ref{table:points-to-improve} erläutert.

\begin{table}[H]
\begin{tabular}{p{5cm} | p{9cm}}
  \textbf{Punkt mit Potential} & \textbf{Beschreibung}
  \\ \hline
  XXX & XXX \\
  
 \end{tabular}
 \caption{Übersicht der Punkte mit Verbesserungspotential}
 \label{table:points-to-improve}
\end{table}

\subsection{Fehlende Punkte für eine praktische Verwendung des Betriebssystems}
Das Betriebssystem weist einige wenige Punkte auf, welche noch nicht implementiert wurden, aber für eine praktische Verwendung fehlen. Tabelle \ref{table:missing-points} zeigt diese Punkte auf.

\begin{table}[H]
\begin{tabular}{p{5cm} | p{9cm}}
  \textbf{Fehlender Punkt} & \textbf{Beschreibung}
  \\ \hline
  ResourceManager & Dieser Manager ist für die Verwaltung von Ressourcen zuständig. Dazu zählen XXX\\
  
 \end{tabular}
 \caption{Übersicht der fehlenden Punkte}
 \label{table:missing-points}
\end{table}

\pagebreak 