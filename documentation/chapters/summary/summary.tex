\section{Zusammenfassung und Ausblick}

In diesem Kapitel werden die erreichten Ergebnisse zusammengefasst. Weiters wird ein Ausblick auf Möglichkeiten der Weiterentwicklung geboten.

\subsection{Zusammenfassung}

Das primäre Ziel dieses Projektes war die Erlangung tiefergehende Kenntnisse in Bezug auf die Systemprogrammierung von Systemen mit beschränkten Ressourcen. Dabei sollten vor allem die theoretische Grundlagen von Betriebssystemen praktisch umgesetzt werden. \\
 
Implementiert und getestet wurde ein Betriebssystem welches sich auch in Langzeittests als stabil erwiesen hat. Das Betriebssystem ist durch die HAL flexibel und ohne größere Aufwände protierbar. Zudem ist es möglich, von einer MicroSD-Karte Applikationen zu laden und auszuführen. 
Bei der Implementierung wurden sämtliche Grundaspekte moderner Betriebssysteme, wie beispielsweise die Interprozesskommunikation oder die virtuelle Speicherverwaltung, behandelt. Zudem wurden die Sicherheitsrisiken durch das saubere Trennen der Adressräume und Benutzermodi stark verringert. \\
 
Es ist zu erwähnen, dass während des Entwicklungsprozesses erwartete wie auch nichterwartete Probleme aufgetreten sind. Diese betreffen in erster Linie Komplikationen, die durch falsches Setzen der Hardwareregister entstanden sind. \\
 
Für das entworfene Betriebssystem wird kein Anspruch auf Vollständigkeit erhoben, da seine Entwicklung agil vorgenommen wurde. So wurden aus Zeitgründen bei der Erstellung der HAL nur die für die vorgesehene DMX-Applikation benötigten Funktionen implementiert. \\
 
Das Betriebssystem ist in der vorliegenden Form voll einsatzbereit und erfüllt alle gesetzten Anforderungen. Daher wird an dieser Stelle seine Entwicklung seitens des Projektteams eingestellt. Unter dem in der Einleitung angegebenen Repository auf GitHub kann es frei zugänglich heruntergeladen werden. \\

\subsection{Ausblick}

Das Betriebssystem schneidet alle Punkte, welche in den Anforderungen erwähnt wurden, zumindest teilweise an.

\subsubsection{Punkte mit Verbesserungspotential}
Einige Punkte des Betriebssystem konnten nicht vollständig abgedeckt werden. Diese Punkte mit Verbesserungspotential werden in Tabelle \ref{table:points-to-improve} erläutert.

\begin{table}[H]
\begin{tabular}{p{5cm} | p{9cm}}
  \textbf{Punkt mit Potential} & \textbf{Beschreibung}
  \\ \hline
  XXX & XXX \\
  
 \end{tabular}
 \caption{Übersicht der Punkte mit Verbesserungspotential}
 \label{table:points-to-improve}
\end{table}

\subsubsection{Fehlende Punkte für eine praktische Verwendung des Betriebssystems}
Das Betriebssystem weist einige wenige Punkte auf, welche noch nicht implementiert wurden, aber für eine praktische Verwendung fehlen. Tabelle \ref{table:missing-points} zeigt diese Punkte auf.

\begin{table}[H]
\begin{tabular}{p{5cm} | p{9cm}}
  \textbf{Fehlender Punkt} & \textbf{Beschreibung}
  \\ \hline
  ResourceManager & Dieser Manager ist für die Verwaltung von Ressourcen zuständig. Dazu zählen XXX\\
  
 \end{tabular}
 \caption{Übersicht der fehlenden Punkte}
 \label{table:missing-points}
\end{table}