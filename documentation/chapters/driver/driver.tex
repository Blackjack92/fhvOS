\section{Treiber}
Die Treiber stellen eine abstrakte Schnittstelle auf den Hardware Abstraction Layer dar. Dadurch muss nicht mehr direkt auf die einzelnen Hardwarekomponenten zugegriffen werden. Ein wesentlicher Aspekt bei der Architektur der Treiber war Abstraktion. Dadurch wird gewährleistet, dass jeder Treiber über die selbe Schnittstelle angesprochen werden kann. Zudem ist die Verwaltung durch den Driver Manager erleichtert.

\subsection{Aufbau eines Treibers}
In Listening \ref{generalDriverInterface} ist die allgemeine Schnittstelle für jeden Treiber ersichtlich. Jeder implementierte Treiber muss diese vorweisen können, ein Beispiel (LED Treiber) dazu ist in Listening \ref{ledDriver} zu sehen.

\lstinputlisting[language=C, caption=Allgemeine Schnittstelle für einen Treiber, label=generalDriverInterface]{chapters/driver/codefiles/generalDriverInterface.c}

\lstinputlisting[language=C, caption=Implementierung der allgemeinen Treiberschnittstelle (LED Beispiel), label=ledDriver]{chapters/driver/codefiles/ledDriver.c}


\pagebreak 